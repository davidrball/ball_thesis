\chapter[Introduction]
{Introduction}
%\begin{figure}
%\centering
%\includegraphics[angle=0,width=\columnwidth]{fig1.pdf}
%\caption[]{}
%\label{fig1}
%\end{figure}
\section{Black Holes and Accretion}

Black holes have drawn substantial attention from both scientists and the public since Karl Schwarzchild first formulated them as a solution to Einstein's field equations (\citealt{schwarzschild1916}).  Since then, scientists have generated a large body of work focused on understanding these captivating objects and the material that collects around them.  In this dissertation, we will focus on understanding the physics of hot ionized gas (or, plasma) that swirls around black holes and heats up to extremely high temperatures, lighting up the direct surroundings of black holes at nearly all wavelengths, from the radio to X-ray.  

Black holes range in size, from just over the mass of the Sun ($3.8 M_{\rm{sun}}$, \citealt{smallbh}), to over a billion times the mass of the Sun ($6.5\times10^9 M_{\rm{sun}}$, e.g., \citealt{eht6}).  We will focus on so-called ``supermassive black holes" (defined for our purposes as $M>10^5 M_{\rm{sun}}$).  Supermassive black holes can have profound impacts on their environments and are responsible for a variety of critical phenomena in terms of galaxy evolution.  For instance, the supermassive black hole at the center of M87 is thought to be responsible for generating powerful jets that we see emerging from a very compact source in this galaxy (e.g., \citealt{sparks1996}).  More recently, this black hole was imaged by the Event Horizon Telescope collaboration (EHT) and showed the telltale sign of the ``black hole shadow" imprinted on the surrounding material's emission (\citealt{eht6}).  By providing the first horizon-scale imaging of black holes, the EHT provides a new testbed not only for General Relativity, but also for our understanding of plasma physics in a regime that terrestrial experiments cannot probe.

The closest supermassive black hole to the Earth is Sagittarius A* (henceforth Sgr~A*), which resides at the center of our own Milky Way galaxy.  Numerous observing campaigns have been dedicated to characterizing the emission from this source, both in terms of its spectrum, as well as its variability (e.g., \citealt{bower2015}).  These detailed observations, when combined with the appropriate physical models, are extremely powerful in furthering our understanding of the physics of these systems.  Sgr~A* is quite dim, and hence, is classified as a ``low-luminosity accretion flow", and is thought to have a very low accretion rate $\sim 3 \times 10^{-5} \; \rm{M_{sun}} \; \rm{yr}^{-1}$ (e.g. \citealt{quataert1999b}).  We will elaborate further on the physics of low-luminosity accretion flows in section \ref{sec_lowlum}.

Observations of Sgr~A* reveal significant multiwavelength variability (see \citealt{eckart2004, marrone2008, neilsen2013, witzel2013, ponti2015, li2015, wang2016}) from the mm and IR to the X-rays.  At high energies, in particular, \citet{neilsen2013} analyzed 3 million seconds of Chandra data dedicated to characterizing both short and long-term X-ray variability of Sgr~A*.  They found that the length of flares varies from a few hundred seconds to 8 ks, with luminosities from $\sim 10^{34} \; \rm{erg} \; \rm{s}^{-1}$ to $2 \times 10^{35} \; \rm{erg} \;  \rm{s}^{-1}$.  More recently, \citet{haggard2019} quantified the behavior of some of the brightest X-ray flares, and showed that some of them have distinct double-peaked structures to the lightcurve.  \citet{eckart2004} carried out simultaneous observations in both the X-ray and near IR, and found that X-ray flares always have a coincident IR flare.  In contrast, there are numerous IR flares without X-ray counterparts.  These flares can offer unique insight into the particle energetics of the accretion flow; highly energetic stochastic flares point towards mechanisms such as shocks and magnetic reconnection, which are capable of quickly producing large numbers of high energy non-thermal particles which significantly alter the observational signatures of the accretion flow.

\section{Low-Luminosity Accretion Flows}
Many supermassive black holes, including Sgr~A*, have small accretion rates, and as a result, have very low luminosities.  In these systems, the radiative efficiency is low due to the low density of the plasma in the flow, and hence they are often coined ``radiatively inefficient accretion flows" (henceforth, RIAFs).  In contrast with relatively cold and radiatively efficient thin disks, RIAFs remain hot because the energy carried away by radiation is negligible compared to the viscously dissipated energy in the system.  The high temperatures and low densities in these disks cause them to be geometrically thick and optically thin (they resemble more of a puffy doughnut than a disk).  As such, their spectra are dominated by various emission processes such as synchrotron and bremsstrahlung that occur at different locations throughout the disk, resulting in spectra that deviate substantially from a simple blackbody.

Due to the low densities in RIAFs, the timescale for electron-ion Coulomb collisions to occur is longer than the heating timescale, which means that the electrons and ions in the flow are energetically decoupled, i.e., they can maintain different temperatures without equilibrating via collisions.  Electrons are expected to be substantially colder than ions for a number of reasons: first, heating mechanisms such as the dissipation of turbulent (or magnetic) energy and shocks tend to favor the more massive species (\citealt{howes2010}, \citealt{rowan2017}) across a broad range of plasma parameters (with the exception of when the plasma-$\beta$, the ratio of gas to magnetic pressure, is very low), second, electrons radiate away their energy at a much faster rate than ions.  In addition to ions and electrons being thermally decoupled, electron-electron and ion-ion Coulomb collision timescales are also longer than dynamical timescales, meaning the distribution function of each species may be highly nonthermal (i.e., deviate substantially from a Maxwellian).

The role of non-thermal electron energy distributions have been addressed in the context of stationary hydrodynamic models by \citet{mahadevan1998}, \citet{ozel2000}, and \citet{yuan2003}, who showed that even a relatively small number of power-law electrons can significantly impact the spectra predicted from a model, generating X-ray power-law tails as well as boosting the low frequency radio flux.  These studies used analytical steady-state solutions to calculate spectra, which do not capture short timescale turbulent effects that play a large role in determining the variability properties of the system.  

The production of non-thermal electrons has been studied extensively with applications not only to active galactic nuclei (AGN), but also pulsar wind nebulae (see \citealt{sironi2017} for a recent review), gamma ray bursts (e.g., \citealt{werner2017}), and the Sun (\citealt{shibata2011}).  These studies usually focus on either magnetic reconnection or shocks as the drivers of nonthermal particle acceleration, but more recent studies have shown that particle acceleration can also be a generic byproduct in the dissipation of turbulence (e.g., \citealt{zhdankin2019, comisso2018, comisso2019}).  Because these processes are present in RIAFs, and the low-collisionality environment can support robust non-thermal distributions, we indeed expect non-thermal electrons to be ubiquitous throughout these systems, with properties (such as the energy distribution) dictated by the local properties in the flow and heating and acceleration processes therein.
%In section \ref{phenom_model}, we will elaborate further on the work that has been done to address the impact of nonthermal electrons in low-luminosity accretion flows, and describe a phenomenological model we developed to explain the X-ray flares from Sgr~A* in the framework of magnetohydrodynamic simulations and radiative transfer calculations.

\label{sec_lowlum}
\section{Magnetohydrodynamics}
While substantial work has been done to analytically understand the steady-state global solutions to RIAFs (see \citealt{yuan2014} for a recent review), these calculations make a number of simplifying assumptions.  In particular, in order to make the calculations tractable, most studies assume steady and axisymmetric accretion flows and also prescribe an artificial kinematic viscosity to enable accretion.  While these solutions are enlightening and highlight some of the basic underlying physics of accretion disks, they are not suited for studying the nonlinear physics of accretion, the turbulence that arises, and the associated variability in emission.  In order to understand these processes, we must turn to numerical simulations.

The most commonly used type of simulation to study accretion flows around black holes are ``general relativistic magnetohydrodynamic" (henceforth GRMHD) simulations.  In short, these are simulations that capture the dynamics and energetics of an ionized gas in the static but curved spacetime around a black hole.  The framework of MHD makes a number of simplifying assumptions.  First, it assumes that the plasma satisfies the fluid approximation: that the collisional mean-free-path is much less than the scale of interest.  Most MHD simulations also utilize the ``ideal MHD" approximation: that the plasma is infinitely conductive.  In other words, because electrons have such a small mass, they respond extremely rapidly to the presence of electric fields and will immediately short out any electric field in the system, ensuring that the fluid remains electrically neutral.  One consequence of the ideal MHD approximation is that it effectively ``freezes" the magnetic field into the fluid, so that the evolution of the magnetic field is simply described as being advected around with the fluid.  In reality, the ideal MHD approximation drops a number of other higher-order terms to simplify the evolution of the magnetic field, we will now briefly cover how the evolution of the magnetic field evolves in the framework of ideal MHD and the assumptions used to get there.  For clarity, all of the following equations in this section are in the nonrelativistic limit, where the equations lend themselves far more readily toward building physical intuition.  We note, however, that the general principles we discuss also apply to the relativistic analogs.

We begin by considering the generalized Ohm's law:

\begin{equation}
	\bold{E}+\bold{u}\times \bold{B} = 
	\eta \bold{J} + \frac{1}{e n}\left(\bold{J} \times \bold{B} - \nabla \cdot \overleftrightarrow{\bold{P}}_{e} + \frac{m_e}{e}\left[\frac{\partial \bold{J}}{\partial t}+\nabla \cdot (\bold{Ju} + \bold{uJ})\right]\right).
	\label{ohm}
\end{equation}

Here $\bold{E}$, $\bold{B}$, $\bold{u}$, and $\bold{J}$ represent the electric field, magnetic field, fluid velocity (the mass-weighted average velocity, including both species), and current, while $\overleftrightarrow{\bold{P}}_e$ represents the electron pressure tensor.  In this dissertation, bolded quantities represent vectors while double arrows above a bold quantity represent rank-2 tensors.  The remaining quantities: $\eta$, $e$, $n$, and $m_e$ are simply the fluid's resistivity, the charge of a proton, the number density of the fluid, and electron mass.  In this equation, the first term on the right hand side is simply Ohmic dissipation. The first term in parenthesis is the so-called ``Hall" term and the following two terms include effects from the divergence of the electron pressure tensor and electron inertial effects, all of which contribute to the electric field.  The terms in parenthesis can be shown to be negligible for most astrophysical cases by casting Equation \ref{ohm} in dimensionless form (e.g., \citealt{gourdain2017}), which yields a coefficient on the term in parenthesis that scales as $\delta_i/L$, where $\delta_i$ is the ion inertial length and $L$ is the length scale of interest.  In most astrophysical cases, this coefficient is extremely small; $\delta_i$ is a microphysical scale that describes the dynamics of particles within a plasma, compared to the huge astrophysical macroscales.  For instance, in the accretion flow around the galactic center, $\delta_i$ is of order $\sim 10$s  of meters, whereas the length scale of interest, $GM/c^2$, is of order $\sim 10^9$ meters.  The ideal approximation sets the entire right hand side to 0 (by the argument that the electrical resistivity, $\eta$, is nearly 0, corresponding to an infinite conductivity $\sigma=1/\eta$).  This results in the much simpler expression for the electric field of

\begin{equation}
	\bold{E}_{\rm{ideal}}=-\bold{u}\times \bold{B}.
	\label{ideal_E}
\end{equation} 

This electric field is often referred to as the ``ideal" or ``motional" electric field, and can be understood simply as the electric field induced by moving magnetic fields.  In order to assess the evolution of the magnetic field, we plug the electric field found in \ref{ideal_E} into Faraday's law:

\begin{equation}
	\frac{1}{c}\frac{\partial \bold{B}}{\partial t} = - \nabla \times \bold{E_{\rm{ideal}}}=\nabla \times (\bold{u}\times\bold{B})
	\label{faraday}
\end{equation}

This leads to a phenomenon known as ``flux freezing": it is relatively straightforward to show that Equation \ref{faraday} implies that the magnetic flux through a fluid element is conserved as the fluid moves.  Another related, although not identical implication is the so-called ``Lundquist Theorem" that shows that Equation \ref{faraday} more generally describes how a any vector field (in this case, $\bold{B}$) is advected by a velocity field.  Under this motion, the magnetic field can be stretched and compressed but cannot reconnect or dissipate.

The underlying assumptions that lead to the equations of ideal MHD make the system of equations much more tractable to implement numerically.  Furthermore, the assumptions are quite sound for a large number of astrophysical systems and scales.  MHD (and GRMHD, when necessary) has proven to be a powerful tool over the last few decades and has explained a wide variety of astrophysical phenomena, including but not limited to: accretion on to black holes, plasma in the outskirts of galaxy clusters, protoplanetary disks, star forming regions, and also the solar wind and corona.

\subsection{Calculating Observables from GRMHD}
Because the systems we are interested in are radiatively inefficient, the GRMHD simulations do not need to account for how photons carry energy away from the system (although much work has been done to include radiation into GRMHD for other purposes).  While the GRMHD simulations describe the three-dimensional distribution of various fluid quantities such as the densities, temperatures, and magnetic field strengths and directions, they do not tell us what the radiation from the system looks like, either in terms of its spectrum or image.  In order to understand the observable properties, we post-process the GRMHD simulations with radiative transfer calculations that take into account the warped paths that light takes through the curved spacetime around the black hole.

These calculations begin by setting up and ``observing screen", effectively a 2D array of points some distance $D$ away from the system we are simulating the observation of.  We then integrate the geodesic equation backwards to calculate the trajectories of rays from the observing plane, towards the source.  Along the geodesic, we integrate the radiative transfer equation, and in principle can include the well-defined emissivities and opacities from a number of emission mechanisms, including, but not limited to, (both thermal and non-thermal) bremsstrahlung, synchrotron, and inverse Compton emission.  In this way, each optical path being integrated corresponds to a pixel on the image plane, and has an intensity associated with it (which can be a function of wavelength), and the collection of these optical paths forms an image. 

In this dissertation, I use the GPU accelerated general-relativistic raytracing and radiative transfer code, GRay (\citealt{chan2013}), which has been an invaluable tool in the speed ups it offers over CPU-based counterparts, enabling high-cadence multiwavelength model development, testing, and comparison.

\subsection{Limitations of the MHD approximations}
While the fundamental assumptions of ideal MHD may be valid on the largest scales in the system, in principle, small regions may arise that violate the underlying assumptions.  For instance, if we consider the generalized Ohm's law (Equation \ref{ohm}) in the limit where the Hall, electron pressure tensor, and electron inertia terms are negligible, but retain Ohmic dissipation, we get the ``resistive Ohm's law",

\begin{equation}
\bold{E} + \bold{u} \times \bold{B} = \eta \bold{J}.
\end{equation}

In order to assess the importance of a finite resistivity on the evolution of the magnetic field, we once again plug the electric field into Faraday's law:


\begin{equation}
	\frac{1}{c}\frac{\partial \bold{B}}{\partial t} =\nabla \times (\bold{u}\times\bold{B}) -\nabla \times \eta \bold{J}
	\label{faraday_resistive}
\end{equation}
	
In the limit of nonrelativistic MHD, the current is simply the curl of the magnetic field, i.e.,

\begin{equation}
	\bold{J} = \frac{c}{4\pi}\nabla \times \bold{B}.
\end{equation}
As such, we can write the evolution of the magnetic field in terms of itself and the fluid velocity as 

\begin{equation}
	\frac{1}{c}\frac{\partial \bold{B}}{\partial t} =\nabla \times (\bold{u}\times\bold{B}) -\nabla^2 \eta \bold{B}.
\end{equation}
In most cases, we assume that the resistivity is roughly constant in space so that it simply acts as a diffusion coefficient via

\begin{equation}
		\frac{1}{c}\frac{\partial \bold{B}}{\partial t} =\nabla \times (\bold{u}\times\bold{B}) -\eta \nabla^2 \bold{B}.
\end{equation}

We see that the effect of a finite resistivity serves to dissipate the magnetic field.  In the ideal MHD assumption, we dropped the resistive term, arguing that $\eta$ is small.  Here, however, we see that even in the limit where $\eta$ is small, if regions arise where $\nabla^2 \bold{B}$ is large, then the $\eta \nabla^2 \bold{B}$ term may end up being dynamically and energetically important to this system.  One such configuration that may arise is a so-called ``current sheet", where the magnetic field reverses its direction over a short length scale.  This configuration is unstable and prone to undergo magnetic reconnection, which we will describe in more detail in Section \ref{sec:mag_recon} and will be a focus of this dissertation.

Another limitation of the MHD framework is the fundamental fluid assumption: that the smallest scale in the system is the collisional mean-free-path.  RIAFs, however, have very low densities and hence collisions between particles are rare.  In this case, one can make reasonable arguments that the overall behavior of the system will likely be similar in both fluid and kinetic frameworks, stemming from the fact that the dispersion relation for Alfv\'en waves, the building block of MHD turbulence, is remarkably similar in both fluid and kinetic limits, only deviating at and below ion kinetic scales.  Another oft-invoked argument is that microinstabilities may occur in a collisionless plasma that serve to scatter particles in a way reminiscent of collisions in a fluid.  These effective collisions will isotropize the particles and provide a means for energy exchange and equilibration.  Numerous studies have more rigorously investigated how kinetic effects may play a role in accretion flows.  Some studies take the approach of adding layers to MHD to incorporate kinetic effects related to anisotropic electron conduction (\citealt{ressler2015, ressler2016}), and velocity space and pressure anisotropies (\citealt{sharma2006}).  More recently, (\citealt{riquelme2012, kunz2016}) have performed fully kinetic particle-in-cell, and hybrid-kinetic simulations of shearing boxes, and found that the most salient features of the magnetorotational instability and associated turbulence from MHD are reproduced, suggesting that MHD may suffice for simulating these systems at large scales.

\section{Magnetic Reconnection}
\label{sec:mag_recon}
Magnetic reconnection is a fundamental plasma process in which opposing field lines in a plasma that are separated by a short distance rearrange their topology, connecting with one another.  In doing so, they release magnetic energy into the ambient plasma in the form of bulk motion, heat, and in some cases, particle acceleration.  Reconnection was first formulated in the context of resistive MHD by Eugene Parker and Peter Sweet in 1956 in an attempt to explain the dynamics of solar flares.  We will now briefly review a dimensional argument that follows their original formulations from \citet{parker1957} and \cite{sweet1958} and is widely known as the ``Sweet-Parker" model of reconnection.  Because this is not a rigorous derivation, but rather a scaling argument, we will not concern ourselves much with factors of order unity, but instead focus on the underlying physics and trends that this model uncovers.

Consider a current sheet in which the x-component of the magnetic field, $B_x$, reverses direction over a distance of $\delta$ in the y-direction, and extends along a distance $L$ in the x-direction.  On either side of the layer, the strength of the magnetic field is $B$.  This geometry will have an associated current in the z-direction with a magnitude of 

\begin{equation}
	J_{\rm{z}} = \frac{c}{4 \pi}\frac{B}{\delta}.
\end{equation}    

Outside of the current sheet, the magnetic field strength is constant, and hence there is no current.  As such, the only electric field in this region is from magnetized plasma flowing towards the layer (in the y-direction), which will have a magnitude of $E_{\rm{outside}} \approx u_{\rm{in}}B$, where $u_{\rm{in}}$ is the inflow speed (the bulk velocity of plasma flowing towards the current layer in the y-direction).  Within the layer, the magnetic field goes to 0, so there is no ideal electric field.  There is, however, a current and finite resistivity.  As such, there will be a resistive electric field with magnitude 

\begin{equation}
E_{\rm{inside}} \approx \frac{c}{4\pi}\frac{\eta B}{\delta}.
\end{equation}

By matching $E_{\rm{inside}}$ to $E_{\rm{outside}}$, we can derive an expression for the inflow speed as a function of the other parameters in the setup,

\begin{equation}
	u_{\rm{in}} = \frac{c}{4 \pi} \frac{\eta B^2}{\delta}.
\end{equation}

In the geometry in which plasma is flowing towards the current layer in the outer region (in the y-direction), and plasma within the current sheet flows out along the x-direction, the equation of energy conservation of energy can be written as

\begin{equation}
	\frac{B^2}{8 \pi} = \frac{1}{2}\rho u_{\rm{out}}^2.
\end{equation}

From here, we see that the outflow speed is simply the Alfv\'en speed: 

\begin{equation}
	u_{\rm{out}}=\frac{B}{\sqrt{4\pi \rho}}
\end{equation}

With a bit algebra, it is straightforward to show that the inflow speed can be written simply as $\eta/\delta$, resulting in a reconnection rate that is far too slow to account for observed astrophysical phenomena.  While the Sweet-Parker model of reconnection is quite simplified and doesn't account for some important and relevant physics, it highlights a few key properties of the geometry and dynamics of current sheets: first, a thin layer of plasma within the current sheet flows outward at the Alfv\`en speed.  Second, the upstream plasma flows in at a rate dictated by the microphysics of resistivity.  If some additional effects occur that can artificially boost the effective resistivity of the plasma in the current layer, then reconnection will speed up and can explain a wide variety of astrophysical phenomena. 

Numerous theoretical models have been developed since the Sweet-Parker model to address the slow reconnection rate of the Sweet-Parker model (see \citealt{loureiro2016} for a comprehensive review of theoretical models of reconnection).  These models vary substantially, from invoking standing slow-mode shocks or background turbulence, to inferring the presence of instabilities in the current sheet that fragment it into numerous magnetic islands (plasmoids), resulting in bursty and fast reconnection, as opposed to the slow and constant reconnection in the Sweet-Parker model.  As we will see later in this dissertation, the so-called ``plasmoid" instability (also often referred to as the tearing instability) plays a crucial role in regulating not only the reconnection rate, but also the particle acceleration properties of reconnection layers.
 

Magnetic reconnection is widely thought to play an important role in the episodic flaring activity of numerous astrophysical systems, including blazar jets (\citealt{giannios2013, petropoulou2016, nalewajko2016}), pulsar wind nebulae (\citealt{coroniti1990, lyubarsky2001, zenitani2001, kirk2003, contopoulos2007, petri2008, sironi2011, cerutti2012,cerutti2014,cerutti2017, philippov2014} see \citealt{sironi2017} for a recent review), gamma ray bursts (\citealt{thompson1994, thompson2006, usov1994, spruit2001, drenkhahn2002, lyutikov2003, giannios2008}). the Sun (\citealt{forbes1996, yokoyama2001, shibata2011}), and accretion flows around black holes (\citealt{galeev1979, dimatteo1998, uzdensky2008, li2015, ball2016, li2017}).  Through magnetic reconnection, energy stored in magnetic fields dissipates into the ambient plasma, resulting in particle heating and, in some cases, particle acceleration.  Electrons accelerated to ultra-relatvistic energies can produce flares and high-energy emission.  Many of these astrophysical systems consist of low-density ``collisionless" plasmas, where the timescale for Coulomb collisions is significantly longer than dynamical timescales.  Here, the dynamics and energetics of magnetic reconnection can be properly captured only by means of a fully-kinetic framework, which can be achieved via numerical techniques such as particle-in-cell (PIC) simulations.


\section{Particle-in-Cell Simulations}
The Particle-in-Cell (henceforth, PIC) method is a fully kinetic framework for simulating a plasma from first principles.  In contrast to a fluid framework, where it is assumed that frequent collisions relax the distribution to an isotropic Maxwellian, in PIC, the distribution function can be arbitrarily complex.  As a concrete example of how this may affect the outcome of a calculation, imagine two electron beams with equal and opposite velocities, such that these two populations of electrons are passing through one another.  In a fluid framework, taking moments over the velocity distribution loses information about the underlying distribution, and the two interpenetrating beams' velocities average to zero.  In a kinetic framework, however, we retain the complex nature of this distribution, which it turns out in this particular example, is important to the overall dynamics: in this setup, the well-known ``Weibel" instability grows, generating small magnetic fields perpendicular to the direction of motion of the beams, and eventually scatters and isotropizes the distribution.  

PIC simulations evolve a large number of particles with continuously varying positions and velocities in a domain that is discretized on a grid.  The grid stores information about electromagnetic fields, which are interpolated (via tri-linear interpolation) to the positions of the particles, to obtain the electric and magnetic fields that each particle is being pushed by.  After updating the positions and velocities of particles, the code deposits the currents from the particles onto the grid, from which it solves Maxwell's equations, and updates the values of the electromagnetic field.  Because PIC simulations self-consistently evolve the electromagnetic fields from the currents and the distribution of particles, they can fully capture the dynamics and energetics of reconnection without any underlying assumptions, making PIC an ideal method for studying magnetic reconnection and the associated particle acceleration.

Due to the finite number of particles on the grid, there is inherent noisiness in PIC simulations that can be mitigated in a few different ways.  First, the code can simply use more particles, resulting in smoother currents and associated electric fields.  Second, the code can implement higher order particle shapes, such that when a particle passes from one grid cell, to the next, the transition is smooth rather than abruptly causing a change in the electromagnetic fields.  Third, the code can apply smoothing filters to the currents before depositing them to the grid, resulting in smoother electromagnetic fields.  While these methods are distinctly different, they tend to result in similar behavior while reducing the noise inherent in the system, and in practice, a combination of them is typically used to optimize performance.

While the PIC method is attractive in that it fully captures the behavior of a collisionless plasma, its main drawbacks are the computational costs and inability to simulate astrophysically large systems due to the separation of scales.  For accuracy and stability, PIC codes must resolve the electron oscillation frequency, $\omega_{pe}=\sqrt{4 \pi e^2 n /m_e}$, where n is the plasma density and $e$ and $m_e$ are the electron charge and mass.  The associated length scale with this frequency is $\delta_e = c/\omega_{pe}$.  For densities typical of the innermost regions of RIAFs, this length scale is of order $\sim 1$ meter, while the astrophysical length scale of interest (for RIAFs, $GM/c^2$) is of order $10^9$ meters.  While computational power limits what PIC can study on large scales, it is an optimal method for studying the detailed physics of small localized regions that occur in larger astrophysical contexts, such as current sheets, shocks, or the dissipation of turbulence.

In this thesis, I employ the fully relativistic 3D PIC code, TRISTAN-MP.  TRISTAN was first developed by \citet{buneman1993}, and then later parallelized using MPI (message passing interface) by \citet{spitkovsky2005}.  While there are multiple options for robustly integrating equations of motion for particles in electromagnetic fields, in my work, I employ TRISTAN-MP's implementation of the standard ``Boris" algorithm that utilizes two half-pushes by the electric field, separated by a rotation about the magnetic field.  I use a version of TRISTAN-MP optimized for studies of reconnection (developed and described in more detail in \citealt{sironi2014}). This simulation employs periodic boundaries along the direction of the current sheet and boundaries that constantly expand at the speed of light in the direction transverse to the current sheet.  The implementation of expanding boundaries allows the simulation to capture the the late-time evolution of reconnection without being limited by the initial size of the domain.  In order to optimize this setup, the boundaries occasionally jump back towards the sheet, so that the simulation does not spend too many resources simulating the upstream inert plasma.  The parallelization is implemented via domain decomposition, with each processor responsible for a slice of the simulation domain across the current sheet.  Because the current sheet is a thin and dense region, by slicing across it, we can efficiently balance the load between different processors in this static way (although, in principle, one can dynamically allocate processors for even better performance, as the structure of the density changes). 

%\subsection{Particle Pushing}
%In principle, one can use a variety of different methods for numerically integrating the equation of motion for particles in a PIC simulation.  In practice, however, some methods turn out much more effective for this particular problem than others.  Most (although not all) PIC simulations utilize the well-known ``Boris push'' algorithm.  In this algorithm, we first apply half the electric field's impulse to the particle, then rotate the particle's position about the magnetic field, and then apply the other half of the electric field's impulse. 
%\subsection{Field Solving}
%\subsection{Interpolation and Particle Shape}


