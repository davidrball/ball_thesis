\chapter[Introduction]
{Introduction}
%\begin{figure}
%\centering
%\includegraphics[angle=0,width=\columnwidth]{fig1.pdf}
%\caption[]{}
%\label{fig1}
%\end{figure}
\section{Black Holes and Accretion}

Black holes have drawn substantial attention from both scientists and the public since Karl Schwarzchild first formulated them as a solution to Einstein's field equations (\citealt{schwarzschild1916}).  Since then, scientists have generated a massive body of work focused on understanding these captivating objects and the material that collects around them.  In this dissertation, we will focus on understanding the physics of hot ionized gas (or, plasma) that swirls around black holes and heats up to extremely high temperatures, lighting up the direct surroundings of black holes at nearly all wavelengths, from the radio to X-ray.  

Black holes range in size, from just over the mass of the Sun ($3.8 M_{\rm{sun}}$, \citealt{smallbh}), to over a billion times the mass of the Sun ($6.5\times10^9 M_{\rm{sun}}$, e.g., \citealt{eht6}).  We will focus on so-called ``supermassive black holes" (defined for our purposes as $M>10^6 M_{\rm{sun}}$).  Supermassive black holes can have profound impacts on their environments and are responsible for a variety of critical phenomena in terms of galaxy evolution.  For instance, the supermassive black hole at the center of M87 is thought to be responsible for generating the powerful jets we see emerging from a very compact source in this galaxy (reference).  More recently, this black hole was imaged by the Event Horizon Telescope collaboration (EHT) and showed the telltale sign of the ``black hole shadow'' imprinted on the surrounding material's hot emission.  By providing the first horizon-scale imaging of black holes, the EHT provides a new testbed not only for General Relativity, but also for our understanding of plasma physics in a regime that terrestrial experiments cannot probe.

The closest supermassive black hole to the Earth is Sagittarius A* (henceforth Sgr~A*), which resides at the center of our own Milky Way galaxy.  Numerous observing campaigns have been dedicated to characterizing the emission from this source, both in terms of its spectrum, as well as its variability (references).  These detailed observations, when combined with the appropriate physical models, are extremely powerful in furthering our understanding of the physics of these systems.  Sgr~A* is quite dim, and hence, is classified as a ``low-luminosity accretion flow'', and is thought to have a very low accretion rate $\sim 3 \times 10^{-5} \; \rm{M_{sun}} \; \rm{yr}^{-1}$ (e.g. \citealt{quataert1999b}).  We will elaborate further on the physics of low-luminosity accretion flows in section \ref{sec_lowlum}.

Observations of Sgr~A* reveal significant multiwavelength variability (see \citealt{eckart2004, marrone2008, neilsen2013, witzel2013, ponti2015, li2015, wang2016}) from the mm and IR to the X-rays.  At high energies, in particular, \citet{neilsen2013} analyzed 3 million seconds of Chandra data dedicated to characterizing both short and long-term X-ray variability of Sgr~A*.  They found that the length of flares varies from a few hundred seconds to 8 ks, with luminosities from $\sim 10^{34} \; \rm{erg} \; \rm{s}^{-1}$ to $2 \times 10^{35} \; \rm{erg} \;  \rm{s}^{-1}$.  More recently, \citet{haggard2019} quantified the behavior of some of the brightest X-ray flares, and showed that some of them have distinct double-peaked structures to the lightcurve.  \citet{eckart2004} carried out simultaneous observations in both the X-ray and near IR, and found that X-ray flares always have a coincident IR flare.  In contrast, there are numerous IR flares without X-ray counterparts.  These flares can offer unique insight into the particle energetics of the accretion flow; highly energetic stochastic flares point towards mechanisms such as shocks and magnetic reconnection, which are capable of quickly producing large numbers of high energy non-thermal particles which significantly alter the observational signatures of the accretion flow.

\section{Low-Luminosity Accretion Flows}
Many supermassive black holes, including Sgr~A*, have small accretion rates, and as a result, have very low luminosities.  In these systems, the radiative efficiency is low due to the low density of the plasma in the flow, and hence are often coined ``radiatively inefficient accretion flows'' (henceforth, RIAFs).  In contrast with relatively cold and radiatively efficient thin disk models, radiatively inefficient accretion flows remain very hot precisely because the energy carried away by radiation is negligible compared to the viscously dissipated energy in the system.  The high temperatures and low densities in these disks cause them to be geometrically thick and optically thin.  As such, their spectra are dominated by various emission processes such as synchrotron and bremsstrahlung that occur at different locations throughout the disk, resulting in spectra that deviate substantially from a simple blackbody.

The low plasma density in RIAFs has a number of interesting implications).  First, the timescale for electron-ion Coulomb collisions to exchange energy is longer than the heating timescale, which means that the electrons and ions likely have different temperatures.  Electrons will be substantially colder than ions for a number of reasons: first, heating mechanisms such as the dissipation of turbulent (or magnetic) energy and shocks tend to favor the more massive species (\citealt{howes2010}, \citealt{rowan2017}) across a broad range of plasma parameters (with the exception of when the plasma-$\beta$, the ratio of gas to magnetic pressure, is very low), second, electrons radiate away their energy at a much faster rate than ions.  Second, and perhaps even more dramatic, is that the electron-electron and ion-ion Coulomb collision timescales are also  longer than dynamical timescales, meaning the distribution function of each species may be highly nonthermal (i.e., deviating substantially from a Maxwellian).

The role of non-thermal electron energy distributions have been addressed in the context of stationary hydrodynamic models by \citet{mahadevan1998}, \citet{ozel2000}, and \citet{yuan2003}, who showed that even a relatively small number of power-law electrons can significantly impact the spectra predicted from a model, generating X-ray power-law tails as well as boosting the low frequency radio flux.  These studies used analytical steady-state solutions to calculate spectra, which do not capture short timescale GRMHD effects that play a large role in determining the variability properties of the system.  In section \ref{phenom_model}, we will elaborate further on the work that has been done to address nonthermal electrons in low-luminosity accretion flows, and describe a phenomenological model we developed to explain the X-ray flares from Sgr~A* in the framework of magnetohydrodynamic simulations and radiative transfer calculations.

\label{sec_lowlum}
\section{Magnetohydrodynamics}
While substantial work has been done to analytically understand the steady-state global solutions to RIAFs (see \citealt{yuan2014} for a recent review), these calculations make a number of simplifying assumptions.  In particular, in order to make the calculations tractable, most studies assume steady and axisymmetric accretion flows and also prescribe an artificial kinematic viscosity to enable accretion.  While these solutions are enlightening and highlight some of the most basic underlying physics of accretion disks, they are not suited for studying the nonlinear physics of accretion, the turbulence that arises, and the associated variability in emission.  In order to understand these processes, we must turn to numerical simulations.

The most commonly used type of simulation to study accretion flows around black holes are ``general relativistic magnetohydrodynamic'' (henceforth GRMHD) simulations.  In short, these are simulations that capture the dynamics and energetics of an ionized gas in the static but curved spacetime around a black hole (GR).  The framework of MHD makes a number of simplifying assumptions itself.  First, it assumes that the plasma satisfies the fluid approximation: that the collisional mean-free-path is much less than the scale of interest.  Most MHD simulations also utilize the ``ideal MHD'' approximation: that the plasma is infinitely conductive.  In other words, because electrons have such a small mass, they respond extremely rapidly to the presence of electric fields and will immediately short out any electric field in the system, ensuring that the fluid remains electrically neutral.  One consequence of the ideal MHD approximation is that it effectively ``freezes'' the magnetic field into the fluid, so that the evolution of the magnetic field is simply described as being advected around with the fluid.  In reality, the ideal MHD approximation drops a number of other higher-order terms to simplify the evolution of the magnetic field, we will now briefly cover how the evolution of the magnetic field evolves in the framework of ideal MHD and the assumptions used to get there.  For clarity, all of the following equations in this section are in the nonrelativistic limit, where the equations lend themselves far more readily toward building physical intuition.  We note, however, that the general principles we discuss also apply to the relativistic analogs.

We will begin by considering the generalized Ohm's law:

\begin{equation}
	\bold{E}+\bold{u}\times \bold{B} = 
	\eta \bold{J} + \frac{1}{e n}\left(\bold{J} \times \bold{B} - \nabla \cdot \overleftrightarrow{\bold{P}}_{e} + \frac{m_e}{e}\left[\frac{\partial \bold{J}}{\partial t}+\nabla \cdot (\bold{Ju} + \bold{uJ})\right]\right).
	\label{ohm}
\end{equation}

Here $\bold{E}$, $\bold{B}$, $\bold{u}$, and $\bold{J}$ represent the electric field, magnetic field, fluid velocity (the mass-weighted average velocity, including both species), and current, while $\overleftrightarrow{P_e}$ represents the electron pressure tensor.  In this dissertation, bolded quantities represent vectors while double arrows above a bold quantity represent rank-2 tensors.  The remaining quantities: $\eta$, $e$, $n$, and $m_e$ are simply the fluid's resistivity, the charge of a proton, the number density of the fluid, and electron mass.  In this equation, the first term on the right hand side is simply Ohmic dissipation. The first term in parenthesis is the so-called ``Hall'' term and the following two terms include effects from the divergence of the electron pressure tensor and electron inertial effects, all of which contribute to the electric field.  For most purposes, the terms within the parenthesis are neglected due to the small factor of $1/en$, which is typically negligible compared to the other terms in the equation in astrophysical contexts.  The ideal approximation sets the entire right hand side to 0 (by the argument that $1/en$ is very small, and that the electrical resistivity $\eta$ is nearly 0, corresponding to an infinite conductivity $\sigma=1/\eta$).  This results in the much simpler expression for the electric field of

\begin{equation}
	\bold{E}_{\rm{ideal}}=-\bold{u}\times \bold{B}.
	\label{ideal_E}
\end{equation} 

The electric field is often referred to as the ``ideal'' or ``motional'' electric field, that can be understood simply as the electric field induced by moving magnetic fields.  In order to assess the evolution of the magnetic field, we plug the electric field found in \ref{ideal_E} into Faraday's law:

\begin{equation}
	\frac{1}{c}\frac{\partial \bold{B}}{\partial t} = - \nabla \times \bold{E_{\rm{ideal}}}=\nabla \times (\bold{u}\times\bold{B})
	\label{faraday}
\end{equation}

This leads to a phenomenon known as ``flux freezing'': it is relatively straightforward to show that Equation \ref{faraday} implies that the magnetic flux through a fluid element is conserved as the fluid moves.  Another related, although not identical implication is the so-called ``Lundquist Theorem'' that shows that Equation \ref{faraday} more generally describes how a any vector field (in this case, $\bold{B}$) is advected by a velocity field.  This means that the magnetic field can be stretched and compressed by fluid motions, but cannot reconnect or dissipate.

The underlying assumptions that lead to the equations of ideal MHD make the system of equations much more tractable to implement numerically.  Furthermore, the assumptions are quite sound for a large number of astrophysical systems and scales.  MHD (and GRMHD, when necessary) has proven to be a powerful tool over the last few decades and has explained a wide variety of astrophysical phenomena, from accretion on to black holes, to the behavior of plasma in the outskirts of galaxy clusters, to the solar wind and corona.

\subsection{Shortcomings of the MHD approximations}
While the fundamental assumptions of ideal MHD may be valid on the largest scales in the system, in principle, small regions may arise that violate the underlying assumptions.  For instance, if we consider the generalized Ohm's law (Equation \ref{ohm}) in the limit where the Hall, electron pressure tensor, and electron inertia terms are negligible, we get the ``resistive Ohm's law'',

\begin{equation}
\bold{E} + \bold{u} \times \bold{B} = \eta \bold{J}.
\end{equation}

In order to assess the importance of a finite resistivity on the evolution of the magnetic field, we once again plug the electric field into Faraday's law:


\begin{equation}
	\frac{1}{c}\frac{\partial \bold{B}}{\partial t} =\nabla \times (\bold{u}\times\bold{B}) -\nabla \times \eta \bold{J}
	\label{faraday_resistive}
\end{equation}
	
In the limit of nonrelativistic MHD, the current is simply the curl of the magnetic field, i.e.,

\begin{equation}
	\bold{J} = \frac{c}{4\pi}\nabla \times \bold{B}.
\end{equation}
As such, we can write the evolution of the magnetic field in terms of itself and the fluid velocity as 

\begin{equation}
	\frac{1}{c}\frac{\partial \bold{B}}{\partial t} =\nabla \times (\bold{u}\times\bold{B}) -\nabla^2 \eta \bold{B}.
\end{equation}
In most cases, we assume that the resistivity is roughly constant in space so that it simply acts as a diffusion coefficient via

\begin{equation}
		\frac{1}{c}\frac{\partial \bold{B}}{\partial t} =\nabla \times (\bold{u}\times\bold{B}) -\eta \nabla^2 \bold{B}.
\end{equation}

We see that the effect of a finite resistivity serves to dissipate the magnetic field.  In the ideal MHD assumption, we dropped the resistive term, arguing that $\eta$ is small.  Here, however, we see that even in the limit where $\eta$ is small, if regions arise where $\nabla^2 \bold{B}$ is large, then the $\eta \nabla^2 \bold{B}$ term may end up being dynamically and energetically important to this system.  One such type of configuration that may arise is a so-called ``current sheet'', where the magnetic field reverses its direction over a small length scale.  This configuration is unstable and prone to undergo magnetic reconnection, which we will describe in more detail in the following section and will be a focus of this dissertation.

\section{Magnetic Reconnection}
Magnetic reconnection is a fundamental plasma process in which opposing field lines in a plasma that are separated by a short distance rearrange their topology, connecting with one another.  In doing so, they release magnetic energy into the ambient plasma in the form of bulk motion, heat, and in some cases, particle acceleration.  Reconnection was first formulated in the context of resistive MHD by Eugene Parker and Peter Sweet in 1956 in an attempt to explain the dynamics of solar flares.  We will now briefly review a dimensional argument that follows their original formulations from \citet{parker1957} and \cite{sweet1958} and is widely known as the ``Sweet-Parker'' model of reconnection.  Because this is not a rigorous derivation, but rather a scaling argument, we will not concern ourselves much with factors of order unity, but instead focus on the underlying physics and trends that this model uncovers.

Consider a current sheet in which the x-component of the magnetic field, $B_x$, reverses direction over a distance of $\delta$ in the y-direction, along a distance $L$ in the x-direction.  On either side of the layer, the strength of the magnetic field is $B$.  This geometry will have an associated current in the z-direction with a magnitude of 

\begin{equation}
	J_{\rm{z}} = \frac{c}{4 \pi}\frac{B}{\delta}.
\end{equation}    
We break down this problem into two domains: outside of the current sheet and inside the current sheet, and match the electric fields at the boundary. Outside of the current sheet, the magnetic field strength is constant, and hence there is no current.  As such, the only electric field in this region is from magnetized plasma flowing towards the layer (in the y-direction), which will have a magnitude of $E_{\rm{outside}} \approx u_{\rm{in}}B$, where $u_{\rm{in}}$ is the inflow speed (the bulk velocity of plasma flowing towards the current layer in the y-direction).  Within the layer, the magnetic field goes to 0, so there is no ideal electric field.  There is, however, a current and finite resistivity.  As such, there will be a resistive electric field with magnitude 

\begin{equation}
E_{\rm{inside}} \approx \frac{c}{4\pi}\frac{\eta B}{\delta}.
\end{equation}

By matching $E_{\rm{inside}}$ to $E_{\rm{outside}}$, we can derive an expression for the inflow speed as a function of the other parameters in the setup,

\begin{equation}
	u_{\rm{in}} = \frac{c}{4 \pi} \frac{\eta B^2}{\delta}.
\end{equation}

In the geometry in which plasma is flowing towards the current layer in the outer region (in the y-direction), and plasma within the current sheet flows out along the x-direction, the equation of energy conservation of energy can be written as

\begin{equation}
	\frac{B^2}{4 \pi} = \frac{1}{2}\rho u_{\rm{out}}^2.
\end{equation}

From here, we see that the outflow speed is simply the Alfv\'en speed: 

\begin{equation}
	u_{\rm{out}}=\frac{B}{\sqrt{4\pi \rho}}
\end{equation}

Magnetic reconnection is widely thought to play an important role in the episodic flaring activity of numerous astrophysical systems, including blazar jets (\citealt{giannios2013, petropoulou2016, nalewajko2016}), pulsar wind nebulae (\citealt{coroniti1990, lyubarsky2001, zenitani2001, kirk2003, contopoulos2007, petri2008, sironi2011, cerutti2012,cerutti2014,cerutti2017, philippov2014} see \citealt{sironi2017} for a recent review), gamma ray bursts (\citealt{thompson1994, thompson2006, usov1994, spruit2001, drenkhahn2002, lyutikov2003, giannios2008}). the Sun (\citealt{forbes1996, yokoyama2001, shibata2011}), and accretion flows around black holes (\citealt{galeev1979, dimatteo1998, uzdensky2008, li2015, ball2016, li2017}).


\section{Particle-in-Cell Simulations}


